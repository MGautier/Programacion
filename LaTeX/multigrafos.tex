\documentclass[a4paper,12pt]{article}

\usepackage{estilos}

\pagestyle{empty}

\begin{document}

\figuratikz{1}{Multigrafo de 6 vértices}
{
  \tikzstyle{place}=[circle,thick,draw=blue!20,fill=green!20,minimum size=6mm]
  \tikzstyle{texto}=[]

  \begin{scope}

    \node [place] (c1) {$v_1$};

    \node [place] (c2) [above of=c1,xshift=2.75cm,yshift=.75cm] {$v_2$};

    \node [place] (c3) [below of=c2,yshift=-1.5cm,xshift=-.4cm] {$v_3$}
    edge [] (c2)
    edge [] (c1);

    \node [place] (c4) [below of=c3,yshift=-.5cm,xshift=.2cm] {$v_4$};

    \node [place] (c5) [below of=c1,yshift=-2cm,xshift=.2cm] {$v_5$}
    edge [] (c1)
    edge [] (c4);

    \node [place] (c6) [left of=c1,yshift=-1cm,xshift=-.5cm] {$v_6$}
    edge [] (c5)
    edge [] (c1)
    edge [] (c3);


    \draw (c1) .. controls +(-50:2cm) and +(80:1cm) .. (c5);

    \draw (c2) .. controls +(-1:2cm) and +(20:1cm) .. (c4);
    \draw (c2) .. controls +(-50:2cm) and +(80:1cm) .. (c4);


  \end{scope}

}

\end{document}
