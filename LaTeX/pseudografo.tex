\documentclass[a4paper,12pt]{article}

\usepackage{estilos}

\pagestyle{empty}

\begin{document}

\figuratikz{1}{Pseudografo de 5 vértices}
{
  \tikzstyle{place}=[circle,thick,draw=blue!20,fill=green!20,minimum size=6mm]
  \tikzstyle{texto}=[]

  \begin{scope}

    \node [place] (c1) {$v_1$}
    edge [loop above] (c1);

    \node [place] (c2) [above of=c1,xshift=2.75cm,yshift=.75cm] {$v_2$}
    edge [loop above] (c2)
    edge [loop right] (c2);

    \node [place] (c3) [below of=c2,yshift=-1.5cm,xshift=-.4cm] {$v_3$}
    edge [] (c2)
    edge [] (c1);

    \node [texto] (c4) [below of=c3,yshift=-.5cm,xshift=.2cm] {};

    \node [place] (c5) [below of=c1,yshift=-2cm,xshift=.2cm] {$v_4$}
    edge [] (c1);

    \node [place] (c6) [left of=c1,yshift=-1cm,xshift=-.5cm] {$v_5$}
    edge [] (c5)
    edge [] (c1)
    edge [] (c3);


    \draw (c1) .. controls +(-50:2cm) and +(80:1cm) .. (c5);
    \draw (c1) .. controls +(-100:2cm) and +(-180:1cm) .. (c5);
    \draw (c6) .. controls +(-10:2cm) and +(-100:1cm) .. (c3);


  \end{scope}

}
\end{document}
