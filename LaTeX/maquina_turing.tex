\documentclass[a4paper,12pt]{article}

\usepackage{estilos}

\pagestyle{empty}

\begin{document}
\figuratikz{1}{Una máquina de Turing.}
{
  \tikzstyle{texto}=[]

  \draw (1,1) +(-.5,-.5) rectangle ++(.5,.5);
  \draw (1,1) node{$\cdots$};

  \draw (2,1) +(-.5,-.5) rectangle ++(.5,.5);
  \draw (2,1) node{$B$};

  \draw (3,1) +(-.5,-.5) rectangle ++(.5,.5);
  \draw (3,1) node{$B$};

  \draw (4,1) +(-.5,-.5) rectangle ++(.5,.5);
  \draw (4,1) node{$X_1$};

  \draw (5,1) +(-.5,-.5) rectangle ++(.5,.5);
  \draw (5,1) node{$X_2$};

  \draw (6,1) +(-.5,-.5) rectangle ++(2.5,.5);

  \draw (9,1) +(-.5,-.5) rectangle ++(.5,.5);
  \draw (9,1) node (d1) {$X_i$};

  \draw (10,1) +(-.5,-.5) rectangle ++(1.5,.5);

  \draw (12,1) +(-.5,-.5) rectangle ++(.5,.5);
  \draw (12,1) node{$X_n$};

  \draw (13,1) +(-.5,-.5) rectangle ++(.5,.5);
  \draw (13,1) node{$B$};

  \draw (14,1) +(-.5,-.5) rectangle ++(.5,.5);
  \draw (14,1) node{$B$};

  \draw (15,1) +(-.5,-.5) rectangle ++(.5,.5);
  \draw (15,1) node{$\cdots$};

  \draw[ultra thick,white] (16,1) +(0,-.5) rectangle ++(-.5,.5);
  \draw[ultra thick,white] (0,1) +(0,-.5) rectangle ++(.5,.5);

  \draw (8,3) +(0,0) rectangle ++(2,2);

  \node [texto] (t1) [above of=d1,yshift=2cm] {de};
  \node [texto] (t2) [above of=t1,yshift=-.5cm] {Unidad};
  \node [texto] (t3) [below of=t1,yshift=.5cm] {Control};

  \draw[post] (8.75,3)   .. controls +(50:-2cm) and +(-100:-1cm) .. (9,1.5);

}
\end{document}
