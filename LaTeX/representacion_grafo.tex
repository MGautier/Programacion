\documentclass[a4paper,12pt]{article}

\usepackage{estilos}

\pagestyle{empty}

\begin{document}

\figuratikz{1}{Representación gráfica de un grafo}
{
  \tikzstyle{place}=[circle,thick,draw=blue!20,fill=green!20,minimum size=6mm]
  \tikzstyle{texto}=[]

  \begin{scope}

    \node [place] (c1) {$v_1$};

    \node [place] (c2) [above of=c1,xshift=-2.75cm,yshift=1.75cm] {$v_2$};

    \node [place] (c3) [left of=c2,xshift=-1.4cm] {$v_3$};

    \node [place] (c5) [left of=c1,xshift=-2cm] {$v_5$};

    \node [place] (c4) [left of=c5,xshift=-2cm] {$v_4$};

    %en la sentencia draw junto a su formato controls tenemos dos parámetros
    % antes del "and" y después de este. El segundo (x:estecm) tira la flecha hacia la derecha y el primero la tira hacia abajo

    \draw (c1) .. controls +(-150:2cm) and +(10:1cm) .. (c2);
    \draw (c2) .. controls +(-150:2cm) and +(40:1cm) .. (c5);
    \draw (c3) .. controls +(-80:-1cm) and +(5:-1cm) .. (c2);
    \draw (c3) .. controls +(-10:1cm) and +(-150:2cm) .. (c5);
    \draw (c3) .. controls +(10:-2cm) and +(10:2cm) .. (c4);


  \end{scope}

}

\end{document}
