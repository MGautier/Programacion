\documentclass[a4paper,12pt]{article}

\usepackage{estilos}

\pagestyle{empty}

\begin{document}
\figuratikz{1}{Un DAG con un solo orden topológico $(G,A,B,C,F,E,D)$}
{
  \tikzstyle{place}=[circle,thick,draw=green!20,fill=blue!20,minimum size=4mm]

  \begin{scope}

    \node [place] (c1) [label=left:\textcolor{black}{$A$}] {};

    \node [place] (c2) [below of=c1,yshift=-1cm,label=left:\textcolor{black}{$G$}] {}
    edge [post] (c1);

    \node [place] (c3) [right of=c1,xshift=.5cm,yshift=.5cm,label=above:\textcolor{black}{$B$}] {}
    edge [pre] (c1);

    \node [place] (c4) [below of=c3,yshift=-.2cm,xshift=.4cm,label=below:\textcolor{black}{$C$}] {}
    edge [pre] (c3)
    edge [pre] (c1);

    \node [place] (c5) [right of=c2,xshift=3cm,label=right:\textcolor{black}{$F$}] {}
    edge [pre] (c2)
    edge [pre] (c4);

    \node [place] (c6) [right of=c4,xshift=1cm,yshift=.3cm,label=right:\textcolor{black}{$E$}] {}
    edge [pre] (c4)
    edge [pre] (c5);

    \node [place] (c7) [right of=c6,xshift=.4cm,yshift=.8cm,label=above:\textcolor{black}{$D$}] {}
    edge [pre] (c3)
    edge [pre] (c6);


  \end{scope}

}
\end{document}
